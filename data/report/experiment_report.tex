\documentclass[sigconf]{acmart}

\AtBeginDocument{ \providecommand\BibTeX{ Bib\TeX } }
\setcopyright{acmlicensed}
\copyrightyear{2025}
\acmYear{2025}
\acmDOI{XXXXXXX.XXXXXXX}

\acmConference[BI 2025]{Business Intelligence}{-}{-}

\begin{document}

\title{BI2025 Experiment Report - Group 04}
%% ---Authors: Dynamically added ---

          \author{Emina Skrijelj}
          \authornote{Student A, Matr.Nr.: 12410074}
          \affiliation{
            \institution{TU Wien}
            \country{Austria}
          }
          
          \author{Azra Sisic}
          \authornote{Student B, Matr.Nr.: 12024721}
          \affiliation{
            \institution{TU Wien}
            \country{Austria}
          }
          

\begin{abstract}
  This report documents the machine learning experiment for Group 04, following the CRISP-DM process model.
\end{abstract}

\ccsdesc[500]{Computing methodologies~Machine learning}
\keywords{CRISP-DM, Provenance, Knowledge Graph, Machine Learning}

\maketitle

%% --- 1. Business Understanding ---
\section{Business Understanding}

\subsection{Data Source and Scenario}
The dataset consists of 20,000 samples of handwritten English capital letters,
where each instance represents a scanned image that has been converted 
into 16 numeric features describing shape, stroke patterns, and pixel distributions. 
These features allow the letters to be analyzed without storing the original images.
A scenario for using this data is an automated mail-sorting system in a postal company,
where handwritten characters on envelopes must be recognized quickly and accurately so that
routing and delivery decisions can be made without manual intervention.

\subsection{Business Objectives}
The main business objective is to support automated mail processing by developing a model 
that can recognize handwritten letters without human involvement. By improving the speed 
and accuracy of character recognition, the company aims to reduce manual sorting costs, decrease 
processing time per mail item, and minimize sorting errors that lead to delays.

%% --- 2. Data Understanding ---
\section{Data Understanding}
\textbf{Dataset Description:} 

The following features were identified in the dataset:

\begin{table}[h]
  \caption{Raw Data Features}
  \label{tab:features}
  \begin{tabular}{lp{0.2\linewidth}p{0.4\linewidth}}
    \toprule
    \textbf{Feature Name} & \textbf{Data Type} & \textbf{Description} \\
    \midrule
    
    \bottomrule
  \end{tabular}
\end{table}

%% --- 3. Data Preparation ---
\section{Data Preparation}
\subsection{Data Cleaning}
Describe your Data preparation steps here and include respective graph data.


%% --- 4. Modeling ---
\section{Modeling}

\subsection{Hyperparameter Configuration}
The model was trained using the following hyperparameter settings:

\begin{table}[h]
  \caption{Hyperparameter Settings}
  \label{tab:hyperparams}
  \begin{tabular}{lp{0.4\linewidth}l}
    \toprule
    \textbf{Parameter} & \textbf{Description} & \textbf{Value} \\
    \midrule
    
    \bottomrule
  \end{tabular}
\end{table}

\subsection{Training Run}
A training run was executed with the following characteristics:
\begin{itemize}
    \item \textbf{Algorithm:} 
    \item \textbf{Start Time:} 
    \item \textbf{End Time:} 
    \item \textbf{Result:}  = 
\end{itemize}

%% --- 5. Evaluation ---
\section{Evaluation}

%% --- 6. Deployment ---
\section{Deployment}

\section{Conclusion}

\end{document}
