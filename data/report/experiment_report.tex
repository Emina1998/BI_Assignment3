\documentclass[sigconf]{acmart}

\AtBeginDocument{ \providecommand\BibTeX{ Bib\TeX } }
\setcopyright{acmlicensed}
\copyrightyear{2025}
\acmYear{2025}
\acmDOI{XXXXXXX.XXXXXXX}

\acmConference[BI 2025]{Business Intelligence}{-}{-}

\begin{document}

\title{BI2025 Experiment Report - Group 04}
%% ---Authors: Dynamically added ---

          \author{Emina Skrijelj}
          \authornote{Student A, Matr.Nr.: 12410074}
          \affiliation{
            \institution{TU Wien}
            \country{Austria}
          }
          
          \author{Azra Sisic}
          \authornote{Student B, Matr.Nr.: 12024721}
          \affiliation{
            \institution{TU Wien}
            \country{Austria}
          }
          

\begin{abstract}
  This report documents the machine learning experiment for Group 04, following the CRISP-DM process model.
\end{abstract}

\ccsdesc[500]{Computing methodologies~Machine learning}
\keywords{CRISP-DM, Provenance, Knowledge Graph, Machine Learning}

\maketitle

%% --- 1. Business Understanding ---
\section{Business Understanding}

\subsection{Data Source and Scenario}
The dataset consists of 20,000 samples of handwritten English capital letters,
where each instance represents a scanned image that has been converted 
into 16 numeric features describing shape, stroke patterns, and pixel distributions. 
These features allow the letters to be analyzed without storing the original images.
A scenario for using this data is an automated mail-sorting system in a postal company,
where handwritten characters on envelopes must be recognized quickly and accurately so that
routing and delivery decisions can be made without manual intervention.

\subsection{Business Objectives}
The main business objective is to support automated mail processing by developing a model 
that can recognize handwritten letters without human involvement. By improving the speed 
and accuracy of character recognition, the company aims to reduce manual sorting costs, decrease 
processing time per mail item, and minimize sorting errors that lead to delays.

\subsection{Business Success Criteria}
Business success will be achieved if the final system consistently recognizes 
handwritten letters with high accuracy and operates fast enough for large-scale mail processing. 
A useful threshold for business success is achieving at least around 90\% recognition accuracy on 
previously unseen samples, reducing manual sorting effort by a measurable amount, and making sure that 
the automated system does not introduce operational risks like frequent misclassification of 
visually similar letters. The model must be reliable enough to support continuous deployment in a 
logistics workflow.

\subsection{Data Mining Goals}
The goal is to build a predictive model that can classify each input into one of 26 letter categories
based on the extracted numerical features. This includes selecting a suitable machine learning 
algorithm, performing preprocessing, tuning key hyperparameters, and evaluating 
the model using appropriate metrics for multiclass classification.

\subsection{Data Mining Success Criteria}
The goal is to have a trained model that reaches strong classification 
performance on a test set, ideally approaching or exceeding 90\% accuracy, 
while maintaining balanced performance across all classes. The process should demonstrate 
good improvement through hyperparameter tuning and produce reproducible results documented 
in the provenance graph. Success means that the model meets the technical requirements needed 
to fulfill the business objectives.

\subsection{AI Risk Aspects}
Some of the AI risks still exist even though the dataset does not involve personally sensitive attributes.
The system may misclassify visually similar letters,
which could lead to operational errors in automated sorting. The dataset might represent only a
limited range of handwriting styles, making the model less robust when applied to real-world data
that varies in pen type, writing habits, or scanning quality. Also the system must be monitored
for drift and provide fallback options so that uncertain predictions do not automatically trigger
incorrect routing decisions. Transparency about model limitations is important to guarantee safe deployment.


%% --- 2. Data Understanding ---
\section{Data Understanding}
\textbf{Dataset Description:} 

The following features were identified in the dataset:

\begin{table}[h]
  \caption{Raw Data Features}
  \label{tab:features}
  \begin{tabular}{l l l p{0.35\linewidth}}
    \toprule
    \textbf{Feature Name} & \textbf{Data Type} & \textbf{Unit} & \textbf{Description} \\
    \midrule
    {du_table_rows}
    \bottomrule
  \end{tabular}
\end{table}


%% --- 3. Data Preparation ---
\section{Data Preparation}
\subsection{Data Cleaning}
Describe your Data preparation steps here and include respective graph data.


%% --- 4. Modeling ---
\section{Modeling}

\subsection{Hyperparameter Configuration}
The model was trained using the following hyperparameter settings:

\begin{table}[h]
  \caption{Hyperparameter Settings}
  \label{tab:hyperparams}
  \begin{tabular}{lp{0.4\linewidth}l}
    \toprule
    \textbf{Parameter} & \textbf{Description} & \textbf{Value} \\
    \midrule
    
    \bottomrule
  \end{tabular}
\end{table}

\subsection{Training Run}
A training run was executed with the following characteristics:
\begin{itemize}
    \item \textbf{Algorithm:} 
    \item \textbf{Start Time:} 
    \item \textbf{End Time:} 
    \item \textbf{Result:}  = 
\end{itemize}

%% --- 5. Evaluation ---
\section{Evaluation}

%% --- 6. Deployment ---
\section{Deployment}

\section{Conclusion}

\end{document}
